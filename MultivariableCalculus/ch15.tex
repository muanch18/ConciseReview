\setcounter{chapter}{14}
\chapter{Partial Derivatives}

\section{Functions of Several Variables}
A \textbf{function $f$ of two variables} is a rule that assigns to each ordered pair of real numbers $(x,y)$ in a set $D$ a unique real number denoted by $f(x,y)$. The set $D$ is the \textbf{domain} of $f$ and its \textbf{range} is the set of values f takes on. 
Sometimes it is written as $z = f(x,y)$, where $x$ and $y$ are independent variables and $z$ is the dependent variable.

The \textbf{level curves} of a function $f$ of two variables are the curves with equations $f(x,y) = k$, where $k$ is a cosntant (in the range of $f$).

\subsection*{Example}
Sketch the level curves of the function $f(x,y) = 6 - 3x - 2y$ for the values $k = -6, 0, 6, 12$.
\subsubsection*{Solution}
The level curves are $$6-3x-2y = k\qquad or\qquad 3x + 2y + (k-6) = 0$$. The curves are drawn below, with the graph showing a family of lines with slope $-\frac{3}{2}$.

A \textbf{function of three variables}, $f$, is a rule that assigns to each ordered triple $(x,y,z)$ which can also be written as $T = f(x,y,z)$.

\section{Limits and Continuity}
Let f be a function of two variables whose domain $D$ includes points arbitrarily close to $(a,b)$ along any path that stays within the domain of $f$. 
Then we say that the \textbf{limit of $f(x,y)$ as $(x,y)$} approaches $(a,b)$ is $L$ and we write
$$\lim_{(x,y)\rightarrow (a,b)}f(x,y) = L$$
If $f(x,y)\rightarrow L_1$ as $(x,y)\rightarrow (a,b)$ along a path $C_1$ and $f(x,y)\rightarrow L_2$ as $(x,y)\rightarrow (a,b)$ along a path $C_2$, where $L_1\neq L_2$, then $lim_{(x,y)\rightarrow (a,b)}$ does not exist.

\subsection*{Example}

\section{Partial Derivatives}
If $g$ has a derivative at $a$, then we call it the \textbf{partial derivative of $f$ with respect to $x$ at $(a,b)$} and denoite it by $f_x(a,b)$. Thus
$$f_x(a,b) = g^\prime(a)\qquad where\qquad g(x) = f(x,b)$$.
\subsection*{Notations for Partial Derivatives}
If $z = f(x,y)$, we write $$f_x(x,y) = f_x = \frac{\partial f}{\partial x} = \frac{\partial}{\partial x}f(x,y) = \frac{\partial z}{\partial x} = f_1 = D_1f = D_xf$$
$$f_y(x,y) = f_y = \frac{\partial f}{\partial y} = \frac{\partial}{\partial y}f(x,y) = \frac{\partial z}{\partial y} = f_2 = D_2f = D_yf$$

\subsection*{Example}
If $f(x,y) = x^3 + x^2y^3 - 2y^2$, find $f_x(2,1)$ and $f_y(2,1)$. 
\subsection*{Solution}
Holdying $y$ constant and differentiating with respect to $x$, we get 
$$f_x(x,y) = 3x^2 + 2xy^3$$ and so $$f_x(2,1) = 3\cdot 2^2 + 2\cdot 2\cdot 1^3 = 16$$
Holding $x$ constant and differentiatng with respect to $y$, we get $$f_y(x,y) = 3x^2y^2 - 4y$$ $$f_y(2,1) = 3\cdot 2^2\cdot 1^2 - 4\cdot 1 = 8$$

\subsection*{Higher Derivatives}
If $f$ is a function of two varaibles, then its partial derivatives $f_x$ and $f_y$ are also functions of two variables, so we can consider their partial derivatives $(f_x)_x$, 
$(f_x)_y$, $(f_y)_x$, and $(f_y)_y$ which are called the \textbf{second partial derivatives}

\subsection*{Example}
Find the second partial derivatives of $$f(x,y) = x^3 + x^2y^3 - 2y^2$$
In the previous example, we found that $$f_x(x,y) = 3x^2 + 2xy^3\qquad f_y(x,y) = 3x^2y^2 - 4y$$
Therefore $$f_{xx} = \frac{\partial}{\partial x}(3x^2 + 2xy^3) = 6x = 2y^3\qquad f_{xy} = \frac{\partial}{\partial y}(3x^2 + 2xy^3) = 6xy^2$$
$$f_{yx} = \frac{\partial}{\partial x}(3x^2y^2 - 4y) = 6xy^2\qquad f_{yy} = \frac{\partial}{\partial y}(3x^2y^2 - 4y) = 6x^2y - 4$$

Suppose $f$ is defined on a disk $D$ that contains the point $(a,b)$. If the functions $f_{xy}(a,b) = f_{yx}(a,b)$ are both continuous on $D$, then
$$f_{xy}(a,b) = f_{yx}(a,b)$$
\subsection*{Partial Differential Equations}
\textbf{Laplace's equation} plays a role in problems of heat conduction, fluid flow, and electric potential. Its solutions are called \textbf{harmonic functions}.
$$\frac{\partial^2u}{\partial x^2} + \frac{\partial^2u}{\partial y^2} = 0$$
The \textbf{wave equation} describes the motion of a waveform, which could be an ocean wave, sound wave, a light wave, or a wave traveling along a vibrating string.
$$\frac{\partial^2u}{\partial t^2} = a^2\frac{\partial^2u}{\partial x^2}$$
The Cobb-Douglas production function models the total production $P$ of an economic system as a function of the amount of labor $L$ and the capital investment $K$.
$$P(L,K) = bL^{\alpha} K^{1-\alpha}$$

\section{Tangent Planes and Linear Approximations}
Suppose $f$ has continuous partial derivatives. An equation of the \textbf{tangent plane} to the surface $z = f(x,y)$ at the point $P(x_0, y_0, z_0)$ is 
$$z - z_0 = f_x(x_0, y_0)(x - x_0) + f_y(x_0, y_0)(y-y_0)$$
\subsection*{Example}
Find the tangent plane to the elliptic paraboloid $z = 2x^2 + y^2$ at the point $(1,1,3)$.
\subsubsection*{Solution} Let $f(x,y) = 2x^2 + y^2$. Then $$f_x(x,y) = 4x\qquad f_y(x,y) = 2y$$ $$f_x(1,1) = 4\qquad f_y(1,1) = 2$$
Then the previous formula gives us the equation of the tangent plane at $(1,1,3)$ as 
$$z - 3 = 4(x-1) + 2(y-1)$$ or $$ = 4x + 2y - 3$$
The linear function whose graph is this tangent plane, namely $$L(x,y) = f(a,b) + f_x(a,b)(x-a) + f_y(a,b)(y-b)$$ is called the 
\textbf{linearization} of $f$ at (a,b) and the approximation $$L(x,y)\approx f(a,b) + f_x(a,b)(x-a) + f_y(a,b)(y-b)$$
is called the \textbf{linear approximation} or the \textbf{tangent plane approximation} of $f$ at $(a,b)$
If $z = f(x,y)$, then $f$ is \textbf{differentiable} at $(a,b)$ if $\Delta z$ (\textbf{increment} of z) can be expressed in the form 
$$\Delta z = f_x(a,b)\Delta x + f_y(a,b)\Delta y\epsilon_1\Delta x + \epsilon_2\Delta y$$
If the partial derivatives $f_x$ and $f_y$ exist near $(a,b)$ and are continuous at $(a,b)$, then $f$ is differentiable at $(a,b)$.
\subsection*{Example}
Show that $f(x,y) = xe^{xy}$ is differentiable at $(1,0)$ and find its linearization there. Then use it approximate $f(1.1,-.1)$
\subsubsection*{Solution}
The partial derivatives are $$f_x(x,y) = e^{xy} + xye^{xy}\qquad f_y(x,y) - x^2e^{xy}$$
$$f_x(1,0) = 1\qquad f_y(1,0) = 1$$ Both $f_x$ and $f_y$ are continuous functions, so $f$ is differentiable. The linearization is 
$$L(x,y) = f(1,0) + f_x(1,0)(x-1) + f_y(1,0)(y-0)$$ $$= 1 + 1(x-1) + 1\cdot y = x + y$$
The corresponding linear approximation is $$xe^{xy}\approx x + y$$ or $$f(1.1, -.1)\approx 1.1 - .1 = 1$$
\subsection*{Differentials}
we define the \textbf{differentials} $dx$ and $dy$ to be independent variables. Then the \textbf{total differential} $dz$ is defined by
$$dz = f_x(x,y)dx + f_y(x,y)dy = \frac{\partial z}{\partial x}dx + \frac{\partial z}{\partial y}dy$$

\section{The Chain Rule}
Suppose that $z = f(x,y)$ is a differentiable function of $x$ and $y$, where $x = g(t)$ and $y = h(t)$ are both differentiable functions of $t$. Then $z$ is a differentaible function of $t$ and
$$\frac{dz}{dt} = \frac{\partial f }{\partial x}\frac{dx}{dt} + \frac{\partial f}{\partial y}\frac{dy}{dt}$$
We can also write the Chain Rule in the form $$\frac{dz}{dt} = \frac{\partial z}{\partial x}\frac{dx}{dt} + \frac{\partial z}{\partial y}\frac{dy}{dt}$$

\textbf{Chain Rule[Case 2]} Suppose that $z = f(x,y)$ is a differentiable function of $x$ and $y$, where $x = g(s,t)$ and $y = h(s,t)$ are differentiable functions of $s$ and $t$. Then
$$\frac{\partial z}{\partial s} = \frac{\partial z}{\partial x}\frac{\partial x}{\partial s} + \frac{\partial z}{\partial y}\frac{\partial y}{\partial s}\qquad \frac{\partial z}{\partial t} = \frac{\partial z}{\partial x}\frac{\partial x}{\partial t} + \frac{\partial z}{\partial y}\frac{\partial y}{\partial t}$$
%include the general form too
Write out the Chain Rule for teh case where $w = f(x,y,z,t)$ and $x = x(u,v)$, $y = y(u,v)$, $z = z(u,v)$, and $t = t(u,v)$.
%insert the tree diagram
We can now write the required expressions: 
$$\frac{\partial w}{\partial u} = \frac{\partial w}{\partial x}\frac{\partial x}{\partial u} + \frac{\partial w}{\partial y}\frac{\partial y}{\partial u} + \frac{\partial w}{\partial z}\frac{\partial z}{\partial u} + \frac{\partial w}{\partial t}\frac{\partial t}{\partial u}$$
$$\frac{\partial w}{\partial v} = \frac{\partial w}{\partial x}\frac{\partial x}{\partial v} + \frac{\partial w}{\partial y}\frac{\partial y}{\partial v} + \frac{\partial w}{\partial z}\frac{\partial z}{\partial v} + \frac{\partial w}{\partial t}\frac{\partial t}{\partial v}$$
\subsection*{Implicit Differentiation}
Given the function x and y being both functions of x, $$\frac{\partial F}{\partial x}\frac{dx}{dx} + \frac{\partial F}{\partial y}\frac{dy}{dx} = 0$$
We solve for $dy/dx$ and obtain $$\frac{dy}{dx} = -\frac{\frac{\partial F}{\partial x}}{\frac{\partial F}{\partial y}} = -\frac{F_x}{F_y}$$
giving us the \textbf{Implicit Function Theorem}
\subsection*{Example}
Find $\frac{\partial z}{\partial x}$ and $\frac{\partial z}{\partial y}$ if $x^3 + y^3 + z^3 + 6xyz = 1$.
\subsubsection*{Solution}
Let $F(x,y,z) = x^3 + y^3 + z^3 + 6xyz - 1$. Then we have
$$\frac{\partial z}{\partial x} = -\frac{F_x}{F_z} = -\frac{3x^2 + 6yz}{3z^2 + 6xy} = -\frac{x^2 + 2yz}{z^2 + 2xy}$$
$$\frac{\partial z}{\partial y} = -\frac{F_y}{F_z} = -\frac{3y^2 + 6xz}{3z^2 + 6xy} = -\frac{y^2 + 2xz}{z^2 + 2xy}$$


\section{Directional Derivatives and The Gradient Vector}
The \textbf{directional derivative} of $f$ at $(x_0, y_0)$ in the direction of a unit vector $\vec{u} = \ev{a,b}$ can be defined as 
$$D_uf(x,y) = f_x(x,y)a + f_y(x,y)b$$
\subsection*{Example}
Find the directional derivative $D_uf(x,y)$ if $$f(x,y) = x^3 - 3xy + 4y^2$$ and $u$ is the unit vector given by angle $\theta = \pi/6$. What is $D_uf(1,2)$?
Using the equation gives
$$D_uf(x,y) = f_x(x,y)\cos{\frac{\pi}{6}} + f_y(x,y)\sin{\frac{\pi}{6}}$$
$$= (3x^2 - 3y)\frac{\sqrt{3}}{2} + (-3x + 8y)\frac{1}{2}$$ $$= \frac{1}{2}\left[3\sqrt{3}x^2 - 3x + (8-3\sqrt{3}y)\right]$$
Therefore $$D_uf(1,2) = \frac{1}{2}\left[3\sqrt{3}x^2 - 3x + (8-3\sqrt{3}y)\right] = \frac{13 - 3\sqrt{3}}{2}$$
\subsection*{The Gradient Vector}
If $f$ is a function of three variables $x$, $y$, and $z$ then the \textbf{gradient} of $f$ is the vector function $\nabla f$ defined by 
$$\nabla f(x,y,z) = \ev{f_x(x,y), f_y(x,y), f_z(x,y)} = \ev{f_x, f_y, f_z} = \frac{\partial f}{\partial x}\ihat + \frac{\partial f}{\partial y}\jhat + \frac{\partial f}{\partial z}\khat$$
The relationship between the directional derivative and the gradient vector is described below:
$$D_uf(x,y, z) = \nabla f (x,y, z)\cdot u$$
Suppose $f$ is a differentiable function of two or three variables. The maximum value of the directional derivative $D_uf(x)$ is $|\nabla f(x)|$ and it occurs when $u$ has teh same direction as the gradient vector.\\
\textbf{Tangent planes to the level surface} can be wrriten in the form 
$$F_x(x_0, y_0, z_0)(x - x_0) + F_y(x_0, y_0, z_0)(y - y_0) + F_z(x_0, y_0, z_0)(z - z_0) = 0$$
The \textbf{normal line} to $S$ at $P$ is the line passing through $P$ perpendicular to the tangent plane.
%insert a gradient vector field

\section{Maximum and Minimum Values}
A function of two variables has a \textbf{local maximum} at $(a,b)$ if $f(x,y)\leq f(a,b)$. If $f(x,y)\geq f(a,b)$, then $f(a,b)$ is a \textbf{local minimum value}.\\
If $f$ has a local maximum or minimum at $(a,b)$ and the first-order partial derivatives of $f$ exist there, then $f_x(a,b) = 0$ and $f_y(a,b) = 0$.

Suppose the second partial derivatives of $f$ are continuous on a disk with center $(a,b)$ and suppose that $f_x(a,b) = 0$ and $f_y(a,b) = 0$. Let 
$$\begin{vmatrix}
    f_{xx}&f_{xy}\\f_{yx}&f_{yy}
\end{vmatrix}$$
\begin{enumerate}[a]
    \item If $D > 0$ and $f_{xx}(a,b) > 0$, then $f(a,b)$ is a local minimum.
    \item If $D > 0$ and $f_{xx}(a,b) < 0$, then $f(a,b)$ is a local maximum.
    \item if $D < 0$, the point is known as a \textbf{saddle point} and is neither a max or a min.
    \item If $D = 0$, the test is indeterminate. 
\end{enumerate}
To find the absolute maximum and minimum values of a continuous function $f$ on a closed, bonded set $D$:
\begin{enumerate}
    \item Find the values of $f$ at the critical points of $f$ in $D$.
    \item Find the extreme values of $f$ on the boundary of $D$.
    \item The largest of the values from the first two steps is the absolute maximum value; the smallest of these values is the absolute minimum value.
\end{enumerate}

