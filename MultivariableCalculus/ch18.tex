\setcounter{chapter}{17}
\chapter{Second-Order Differential Equations}

\section{Second-Order Linear equations}
A \textbf{second-order differential equation} has the form
$$P(x)\frac{d^y}{dx^2} + Q(x)\frac{dy}{dx} + R(x)y = G(x)$$

The base auxillary equation can be given in the form
$$ay^{\prime\prime} + by^\prime + cy = 0$$. The equation can also be written as 
$$ar^2+br+c = 0$$
Determinants of the auxillary equation:
\begin{enumerate}
    \item Case 1: $b^2 - 4ac > 0\qquad y = c_1e^{r^1x} + c_2e^{r^2x}$
    \item Case 2: $b^2 - 4ac = 0\qquad y = c_1e^{r^1x} + c_2xe^{r^2x}$
    \item Case 3: $b^2 - 4ac > 0\qquad y = e^{\alpha x}(c_1cos\beta x + c_2sin\beta x)$\\$\alpha = \frac{-b}{(2a)}$ and $\beta = \frac{\sqrt{4ac-b^2}}{(2a)}$
\end{enumerate}
\subsection*{Example}
Solve the equation $y^{\prime\prime} + y^\prime - 6y = 0$ with the initial values $y(0) - 1$, $y^\prime(0) = 0$
\subsection*{Solution}
The auxillary equation is
$$r^2 + r + 6 = (r+3)(r-2) = 0$$ So the solution is
$$y = c_1e^{2x}+c_2e^{-3x}$$ Differentiatng this, we get
$$y^\prime = 2c_1e^{2x} - 3c_2e^{-3x}$$ Satisfying initial conditions
$$y(0) = c_1 + c_2 = 1$$ $$y^\prime(0) = 2c_1 - 3c_2 = 0$$ $$c_1 = \frac{3}{5}\qquad c_2 = \frac{2}{5}$$ $$y = \frac{3}{5}e^{2x} + \frac{2}{5}e^{-3x}$$ 

\section{Nonhomogeneous Linear Equations}
\subsection*{Method of Undetermined Coefficients}
If the differential equation is given in the form $ay^\prime\prime + by^\prime + cy = G(x)$, then the solution is given by
$$y(x) = y_p(x) + y_c(x)$$
where $y_c(x)$ is the general solution and $y_p(x)$ is the particular solution.

\subsection*{Example}
Solve the equation $y^\prime\prime + y^\prime - 2y = x^2$

\subsection*{Solution}
The auxillary equation becomes
$$r^2+r-2 = (r-1)(r+2) = 0$$
which gives a complementary solution of
$$y_c = c_1e^x + c_2e^{-2x}$$ Since $G(x)$ is a polynomial of the 2nd degree, this gives a particular solution
$$y_p(x) = Ax^2 + Bx + C$$
$$y^\prime_p(x) = 2Ax + B$$ $$y^{\prime\prime}_p(x) = 2A$$
Substituting into the original differential equation gives
$$(2A) + (2Ax + B) - 2 (Ax^2 + Bx + C) = x^2$$ $$A = \frac{-1}{2}\qquad B = \frac{-1}{2}\qquad C = \frac{-3}{4}$$
which gives the particular solution
$$y_p(x) = \frac{-1}{2}x^2 - \frac{1}{2}x - \frac{3}{4}$$
so the final solution is
$$y = c_1e^x + c_2e^{-2x} + \frac{-1}{2}x^2 - \frac{1}{2}x - \frac{3}{4}$$

\subsection*{Method of Variation of Parameters}
If we have already solved the homogeneous equation $ay^\prime\prime + by^\prime + cy = 0$ in the form
$$y(x) = c_1y_1(x) + c_2y_2(x)$$We can replace the constants and write this with an arbitrary function $u(x)$

$$y_p(x) = u_1(x)y_1(x) + u_2(x)y_2(x)$$ $$y^\prime(x) = (u^\prime_1y_1 + u^\prime_2y_2) + (u_1y^\prime_1 + u_2y^\prime_2)$$

\section*{Applications of Second-Order Differential Equations}
If we ignore any external resisting forces, then, by Newton's Second Law, 
$$m\frac{d^2x}{dy^2} = -kx$$ The general solution can be written as
$$x(t) = c_1cos(wt) + c_2sin(wt)$$ or $$x(t) = Acos(wt)+\delta$$ where
$$w = \sqrt{k/m}\\A = \sqrt{c^2_1 + c^2_2}\\cos\delta = \frac{c_1}{A}\qquad sin\delta = \frac{-c_2}{A}$$\\
Dampened vibrations follow the form:
$$m\frac{d^2x}{dy^2} + c\frac{dy}{dx} + kx = 0$$
The roots of the equation follow different cases and show different things about the damping.
\begin{enumerate}
    \item Case 1: $c^2 - 4mk > 0$ (overdamping)
    \item Case 2: $c^2 - 4mk = 0$ (critical damping)
    \item Case 3: $c^3 - 4mk < 0$ (underdamping)
\end{enumerate}
In electrical circuits, a second-order differential equation that relates resistors, inductors and conductors and their sum being equal to the supplied voltage is as follows:
$$L\frac{d^2Q}{dt^2} + R\frac{dQ}{dt} + Q/C = E(t)$$

\section{Series Solution}
When differential equations can not be applied explicity, we use the method of power series; in the solution of the form\\
$$y = f(x) = \sum^\infty_{n=0}c_nx^n = c+0 + c_1x + c_2x^2 + \dots$$

\subsection*{Example}
Use the power series to solve the equation $y^{\prime\prime} + y = 0$
\subsection*{Solution}
Differentiating the power series above gives
$$y^\prime = \sum^\infty_{n=1}nc_nx^{n-1} = c_1 + 2c_2x + \dots$$
$$y^\prime\prime = \sum^\infty_{n=2}n(n-1)c_nx^{n-2} = 2c_2x + 6c_3x + \dots$$
Substituting expressions...$$\sum^\infty_{n = 0}((n+2)(n+1)c_{n+1}+c_n)x^n = 0$$
The corresponding coefficients must be equal to
$$c_{n+2} = \frac{-c_0}{((n+1)(n+2))}$$
By plugging in values for n, the pattern disovered shows\\
\textbf{For even coefficients}: $c_{2n} = \frac{(-1)^nc_)}{(2n)!}$
\textbf{For odd coefficients}: $c_{2n+1} = \frac{(-1)^nc_1}{(2n+1)!}$ 
