\setcounter{chapter}{1}
\chapter{Systems of Linear Equations}

\section{Introduction to Systems of Linear Equations}
Recall that the general equation of a line in $\mathbb{R}^2$ is of the form
$$ax+by = c$$ and the general equation of a plane $\mathbb{R}^3$ is of the form
$$ax+by+cz = d$$ Equations of this form called \textbf{linear equations}.\\

A \textbf{linear equation} in the \textit{n} variables $x_1, x_2, \dots, x_n$ is an equation that can be written in the form
$$a_1x_1+a_2x_2+\dots+a_nx_n = b$$ where the \textbf{coefficient} and the \textbf{constant term} b are constants.

\subsection*{Example}
Solve the system
$$x - y - z = 2$$
$$\qquad y+ + 3z = 5$$
$$\qquad\qquad 5z = 10$$

\subsection*{Solution}
Using the bottom equation, we get $z = 2$, and then plug it into the middle equation. From $y+6 = -5\rightarrow y = -1$. From the first equation, $x = 3$.

\section{Direct Methods for Solving Linear Systems}
The \textbf{coefficient matrix} contains the coefficients of the variables, and the \textbf{augmented matrix} is the coefficient matrix augmented by an extra column containing the constant terms.
A matrix is in \textbf{row echelon form} if it satisfies the following properties:
\begin{enumerate}
    \item Any rows consisting entirely of zeroes are at the bottom.
    \item In each nonzero row, the first nonzero entry(called the \textbf{leading entry}) is in a column to the left of any leading entries below it. 
\end{enumerate}

\subsection*{Gaussian Elimination}
When row reduction is applied to the augmented matrix of a system of a linear equations, we create an equivalent system that can be solved by back substitution. The entire process is known as \textbf{Gaussian elimination}.
\begin{enumerate}
    \item Write the augmented matrix of the system of linear equations.
    \item Use elementary row operations to reduce the augmented matrix to row echelon form. 
    \item Using back subsitution, solve the equivalent system that corresponds to the row-reduced matrix.
\end{enumerate}
The \textbf{rank} of a matrix is the number of nonzero rows in its row echelon form.
\subsection*{Example}
Solve the system:
$$x_1 - x_2 + 2x_3 = 3$$
$$x_1 + 2x_2 - x_3 = -3$$
$$\qquad 2x_2 - 2x_3 = 1$$
\subsection*{Solution}

\subsection*{Gauss-Jordan Elimination}
Steps for this: \begin{enumerate}
    \item Write the augmented matrix of the system of linear equations.
    \item Use elementary row operations to reduce the augmented matrix to row echelon form.
    \item If the resulting system is consistent, solve for the leading variables in terms of any remaining free variables.
\end{enumerate}
A matrix is in \textbf{reduced row echelon form}  if it satisfies the following properties:
\begin{enumerate}
    \item It is in row echelon form
    \item The leading entry in each nonzero row is a 1(called a leading 1).
    \item Each column containing a leading 1 has zeroes everywhere else.
\end{enumerate}
\subsection*{Example}
Determine whether the lines $x = p + su$ and $x = q+tv$ intersect and, ifso find their point of intersection when
$$p = \begin{bmatrix}
    1\\0\\-1
\end{bmatrix}, q = \begin{bmatrix}
    0\\2\\1
\end{bmatrix}, u = \begin{bmatrix}
    1\\1\\1
\end{bmatrix}, and v = \begin{bmatrix}
    3\\-1\\-1
\end{bmatrix}$$
\subsection*{Solution}
$x = p+su = q+tv$, From this, we find out that $s = \frac{5}{4}, t = \frac{3}{4}$. The point of intersection is therefore
$$\begin{bmatrix}
    x\\y\\z
\end{bmatrix} = \begin{bmatrix}
    1\\0\\-1
\end{bmatrix}+\frac{5}{4}\begin{bmatrix}
    1\\1\\1
\end{bmatrix} = \begin{bmatrix}
    \frac{9}{4}\\\frac{5}{4}\\\frac{1}{4}
\end{bmatrix}$$
A system of linear equations is called \textbf{homogeneous} if the constant term in each equation is zero.

\section{Spanning Sets and Linear Independence}
Is the vector $\begin{bmatrix}
    1\\2\\3
\end{bmatrix}$ a linear combination of the vectors $\begin{bmatrix}
    1\\0\\3
\end{bmatrix}$ and $\begin{bmatrix}
    -1\\1\\3
\end{bmatrix}$
\subsection*{Solution}
We want to find scalars $x$ and $y$ such that\\ 
x$\begin{bmatrix}
    1\\0\\3
\end{bmatrix} + y\begin{bmatrix}
    -1\\1\\-3
\end{bmatrix} = \begin{bmatrix}
    1\\2\\3
\end{bmatrix}$.\\ Expanding, we obtain the system
$$x-y = 1$$ $$y = 2$$ $$3x-3y = 3$$ Whose augmented matrix is $$\begin{bmatrix}
    1&0&3\\0&1&2\\)0&0&0
\end{bmatrix}$$. So the solution is $x=3$, $y=2$.
If $S = \{v_1, v_2, \dots, v_k\}$ is a set of vectors in $\mathbb{R}^n$, then the set of all linear combinations is called the \textbf{span} of $v_1, v_2, \dots, v_k$ and is denoted by $span(v_1, v_2, \dots, v_k)$. If $span(S) = R^n$, then $S$ is called a \textbf{spanning set} for $\mathbb{R}^n$.

\section{Applications}
The combustion of ammonia $(NH_3)$ in oxygen produces nitrogen $(N_2)$ and water. Find a balanced chemical equation for this reaction.
$wNH_3 + xO_2\rightarrow yN_2 + zH_2O$\\
Nitrogen: $w = 2y$\\
Hydrogen: $3w = 2z$\\
Oxygen: $2x = z$\\

$$\begin{bmatrix}
    1&0&-2&0&0\\3&0&0&-2&0\\0&2&0&-1&0
\end{bmatrix}$$ rref$\rightarrow$ $$\begin{bmatrix}
    1&0&0&\frac{-2}{3}&0\\0&1&0&\frac{-1}{2}&0\\0&0&1&\frac{-1}{3}&0
\end{bmatrix}$$
$$w = \frac{2}{3}z, x = \frac{1}{2}z, y = \frac{1}{3}z$$
$$w = 4, x = 3, y = 2, z = 6$$
$$4NH_3+3O_2\rightarrow 2N_2 + 6H_2O$$